% !TEX TS-program = xelatex
% !TEX encoding = UTF-8 Unicode

\documentclass[12pt, letterpaper]{article}

% Set document data
\newcommand{\settitle}{Analysis of Muddy Points Survey \#\VAR{contents.get("quiz_number")}}
\newcommand{\setauthor}{Andrew Hoetker}
\title{\settitle}
\author{\setauthor}

% Math environments
\usepackage{amsmath}
\usepackage{amsfonts}
\usepackage{amssymb}

% Chemical formulas
\usepackage[version=4]{mhchem}
\usepackage{chemfig}

% SI units and friends
\usepackage{siunitx}
\DeclareSIUnit\Fahrenheit{\degree F}
\DeclareSIUnit\pound{lb}
\DeclareSIUnit\poundmass{lb\textsubscript{m}}
\DeclareSIUnit\foot{ft}
\DeclareSIUnit\atm{atm}
\DeclareSIUnit\poise{P}

% Page layout
\usepackage[margin=1in,paper=letterpaper]{geometry}
\usepackage{setspace}

% Header
\usepackage{fancyhdr}
\pagestyle{fancy}
\lhead{\setauthor}
\rhead{\settitle}
\rfoot{\tiny{Generated \VAR{
	contents.get("timestamp")
} using Canvas API}}

% XeLaTeX Fonts
%\usepackage{unicode-math}
%\setmainfont[%
%	BoldFont={STIX2Text-Bold.otf},
%	ItalicFont={STIX2Text-Italic.otf}]{STIX2Text-Regular.otf}

%\setmathfont{STIX2Math.otf}

\usepackage{stix2}

% Bibliography
%\usepackage[backend=biber,style=chem-acs,citestyle=chem-acs]{biblatex}
%\addbibresource{styreneproject.bib}

% Fancy linked references
\usepackage{varioref}
\labelformat{equation}{equation~(#1)}
\labelformat{figure}{figure~#1}
\labelformat{table}{table~#1}
\labelformat{appendix}{appendix~#1}

% Hyperlinks
% These will turn internal cross-references, in-text citations,
% and headings into clickable links in the PDF. The link text
% will still be black.
\usepackage{color}
\usepackage[%
	pdfborder=0,
	colorlinks=true,
	urlcolor=black,
	linkcolor=black,
	citecolor=black]{hyperref}

% Figures and captions
\usepackage{graphicx}
\usepackage{booktabs}
\usepackage{caption}
\captionsetup[figure]{justification=centering,labelfont=it}
\captionsetup[table]{justification=centering,labelfont=it}
\usepackage[section]{placeins}

% Filler text
\usepackage{lipsum}

\setcounter{secnumdepth}{0} % sections are level 1


\begin{document}

\maketitle
\tableofcontents

\newpage
\section{Analysis for Both Sections}
\subsection{Attendance by Section}

This week, \VAR{
	contents
		.get("combined")
		.get("attendance")
		.get("count")
} students earned attendance points by responding to the Muddy Points survey.

\begin{figure}[hbt!]
	\centering
	\includegraphics{\VAR{
		contents
			.get("combined")
			.get("attendance")
			.get("filename")
	}}
	\caption{\VAR{
		contents
			.get("combined")
			.get("attendance")
			.get("title")
	}}
\end{figure}

%- if contents.get("combined").get("attendance").get("notes")
\bigskip
\textbf{Note:} \VAR{contents
                        .get("combined")
                        .get("attendance")
                        .get("notes")
                    }
%- endif

\subsection{Self-Assessed Confusion by Section}
\begin{figure}[hbt!]
	\centering
	\includegraphics{\VAR{
		contents
			.get("combined")
			.get("stacked")
			.get("filename")
	}}
	\caption{\VAR{
		contents
			.get("combined")
			.get("stacked")
			.get("title")
	}}
\end{figure}


%- for instructor in contents["instructors"]
\newpage
\section{Analysis for Dr. \VAR{instructor}}

\subsection{Self-Assessed Confusion}

The students were asked, "\VAR{
	contents
	.get("instructors")
	.get(instructor)
	.get("ranked_confusion")
	.get("title")
	.strip()
}".

\begin{figure}[hbt!]
	\centering
	\includegraphics{\VAR{
		contents
		.get("instructors")
		.get(instructor)
		.get("ranked_confusion")
		.get("filename")
	}}
	\caption{Histogram of self-reported confusion.}
\end{figure}
\FloatBarrier

This week, \VAR{
	contents
		.get("instructors")
		.get(instructor)
		.get("ranked_confusion")
		.get("count")
} students responded to the question in this section.
The average self-reported confusion was \VAR{
	"{:.2f}".format(contents
		.get("instructors")
		.get(instructor)
		.get("ranked_confusion")
		.get("mean_confusion")
	)
}, and the median self-reported confusion was \VAR{
	"{:.2f}".format(contents
		.get("instructors")
		.get(instructor)
		.get("ranked_confusion")
		.get("median_confusion")
	)
}.

\subsection{Confusing and Interesting Topics}

The students were asked to respond with a short answer to, "\VAR{
	contents
	.get("instructors")
	.get(instructor)
	.get("short_response")
	.get("title")
	.strip()
}".


These are some responses from students who reported the highest levels of confusion:

%- for question in contents.get("instructors").get(instructor).get("most_confused")
    %- for k, v in question.items()
        \bigskip
        \noindent\textbf{Confusion level: \VAR{v}} \\
        \begin{quote}
        \textit{\VAR{k}}
        \end{quote}
    %- endfor
%- endfor

\begin{figure}[hbt!]
	\centering
	\includegraphics[scale=0.85]{\VAR{
		contents
		.get("instructors")
		.get(instructor)
		.get("short_response")
		.get("filename")
	}}
	\caption{Wordcloud of short responses.}
\end{figure}
\FloatBarrier

%- endfor

\end{document}
